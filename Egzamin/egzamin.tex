\documentclass[12pt]{article}
\usepackage{polski}
\usepackage[utf8]{inputenc}
\usepackage[T1]{fontenc}
\usepackage{graphicx}
\usepackage{grffile}
\usepackage{longtable}
\usepackage{hyperref}
\usepackage{wrapfig}
\usepackage{rotating}
\usepackage[normalem]{ulem}
\usepackage{amsmath}
\usepackage{textcomp}
\usepackage{amssymb}
\usepackage{capt-of}
\usepackage{amsmath}
\usepackage{mathtools}
\usepackage{geometry}
\usepackage{tikz}
\usepackage{ dsfont }


 \geometry{
 a4paper,
 total={170mm,257mm},
 left=20mm,
 top=20mm,
 }

 \hypersetup{
    colorlinks=true,
    linkcolor=blue,
    filecolor=magenta,
    urlcolor=cyan,
}

\title{\textbf{Algorytmów Kwantowych}}
\author{Wojciech Kubiak}
\date{\today}

\begin{document}

\maketitle

\tableofcontents

\section{Liczby zespolone}
Zbiór liczb zespolonych oznaczamy $\mathds{C}$

\subsection{Postać Algebraiczna}
\begin{itemize}
    \item \textbf{Podstawowe informacje:}
    \begin{center}
        $\alpha = a + bi$ , \quad a, b $\in \mathds{R}$ , \quad i $= \sqrt{-1}$ \\
        Re$(\alpha) =$ a \quad - część rzeczywist \\
        Im$(\alpha) =$ b \quad - część urojona
    \end{center}
    \item \textbf{Operacje $\pmb(\mathds{C}, +, -, *, /)$:}
    \begin{center}
        $\alpha = a + bi$ \quad $\beta = c + di$
    \end{center}
    \begin{itemize}
        \item $(+)$ $\alpha + \beta = (a + bi) + (c + di) = (a + c) + (b + d)i$
        \item $(-)$ $\alpha - \beta = (a + bi) - (c + di) = (a - c) + (b - d)i$
        \item $(*)$ $\alpha * \beta = (a + bi) * (c + di) = ac + adi + cbi + bdi^2 = ac + adi + abi - bd = (ac - bd) + (ad + cb)i$
        \item $(/)$ $\frac{\alpha}{\beta} = \frac{(a+bi)(c-di)}{(c+di)(c-di)} = \frac{ac - adi + abi + bd}{c^2 + b^2} = (\frac{ac + bd}{c^2 + b^2}) + (\frac{bc - ad}{c^2 + b^2})i$
    \end{itemize}
\end{itemize}

\subsection{Postać trygonometryczna}
\begin{itemize}
    \item \textbf{Podstawowe informacje:}
    \begin{tikzpicture}
        \draw[thick,->] (0,0) -- (4.5,0) node[anchor=north west] {Re};
        \draw[thick,->] (0,0) -- (0,4.5) node[anchor=south east] {Im};
        \draw [dashed] (0,3)node{bi } -- (3,3)node {       a + bi} -- (3,0) node{a};


    \end{tikzpicture}
\end{itemize}

\end{document}

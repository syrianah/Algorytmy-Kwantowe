% \documentclass[draft]{beamer}
% \usepackage[size=custom,height=10,width=13,scale=0.7]{beamerposter}
% \usepackage[braket, qm]{qcircuit}
% \usepackage{polski}
% \usepackage[utf8]{inputenc}
% \usepackage[T1]{fontenc}
% \usepackage{graphicx}
% \usepackage{grffile}
% \usepackage{longtable}
% \usepackage{hyperref}
% \usepackage{wrapfig}
% \usepackage{rotating}
% \usepackage[normalem]{ulem}
% \usepackage{amsmath}
% \usepackage{textcomp}
% \usepackage{amssymb}
% \usepackage{capt-of}
% \usepackage{amsmath}
% \usepackage{mathtools}
% \usepackage{geometry}
% \usepackage{tikz}
% \usepackage{ dsfont }
% \pdfmapfile{+sansmathaccent.map}


%  \geometry{
%  a4paper,
%  total={170mm,257mm},
%  left=20mm,
%  top=20mm,
%  }

%  \hypersetup{
%     colorlinks=true,
%     linkcolor=blue,
%     filecolor=magenta,
%     urlcolor=cyan,
% }

% \title{\textbf{Algorytmów Kwantowych}}
% \author{Wojciech Kubiak}
% \date{\today}

% \begin{document}

% \maketitle

% \tableofcontents

% \section{Liczby zespolone}
% Zbiór liczb zespolonych oznaczamy $\mathds{C}$

% \subsection{Postać Algebraiczna}
% \begin{itemize}
%     \item \textbf{Podstawowe informacje:}
%     \begin{center}
%         $\alpha = a + bi$ , \quad a, b $\in \mathds{R}$ , \quad i $= \sqrt{-1}$ \\
%         Re$(\alpha) =$ a \quad - część rzeczywist \\
%         Im$(\alpha) =$ b \quad - część urojona
%     \end{center}
%     \item \textbf{Operacje $\pmb(\mathds{C}, +, -, *, /)$:}
%     \begin{center}
%         $\alpha = a + bi$ \quad $\beta = c + di$
%     \end{center}
%     \begin{itemize}
%         \item $(+)$ $\alpha + \beta = (a + bi) + (c + di) = (a + c) + (b + d)i$
%         \item $(-)$ $\alpha - \beta = (a + bi) - (c + di) = (a - c) + (b - d)i$
%         \item $(*)$ $\alpha * \beta = (a + bi) * (c + di) = ac + adi + cbi + bdi^2 = ac + adi + abi - bd = (ac - bd) + (ad + cb)i$
%         \item $(/)$ $\frac{\alpha}{\beta} = \frac{(a+bi)(c-di)}{(c+di)(c-di)} = \frac{ac - adi + abi + bd}{c^2 + b^2} = (\frac{ac + bd}{c^2 + b^2}) + (\frac{bc - ad}{c^2 + b^2})i$
%     \end{itemize}
% \end{itemize}

% \subsection{Postać trygonometryczna}
% \begin{itemize}
%     \item \textbf{Podstawowe informacje:}
%     \begin{tikzpicture}
%         \draw[thick,->] (0,0) -- (4.5,0) node[anchor=north west] {Re};
%         \draw[thick,->] (0,0) -- (0,4.5) node[anchor=south east] {Im};
%         \draw [dashed] (0,3)node{bi } -- (3,3)node {       a + bi} -- (3,0) node{a};


%     \end{tikzpicture}
% \end{itemize}

% \newpage

% \section{Bramki logiczne dla układów kwantowych}

% \subsection{Bramka Hadamarda (H)}
% $|0\rangle = \frac{|0\rangle + |1\rangle}{\sqrt{2}}$ \\
% $|1\rangle = \frac{|0\rangle - |1\rangle}{\sqrt{2}}$
% \[H = \frac{1}{\sqrt{2}}
% \begin{bmatrix}
%     1 & 1 \\
%     1 & -1
% \end{bmatrix}
% =
% \begin{bmatrix}
%     \frac{1}{\sqrt{2}} & \frac{1}{\sqrt{2}} \\
%     \frac{1}{\sqrt{2}} & -\frac{1}{\sqrt{2}}
% \end{bmatrix}
% \]

% \subsection{Bramka Pauliego-X}
% Czasem nazywana "bit-flip" \\
% $|0\rangle = |1\rangle$ \\
% $|1\rangle = |0\rangle$
% \[X =
% \begin{bmatrix}
%     0 & 1 \\
%     1 & 0
% \end{bmatrix}
% \]

% \subsection{Bramka Pauliego-Y}
% $|0\rangle = i|1\rangle$ \\
% $|1\rangle = -i|0\rangle$
% \[Y =
% \begin{bmatrix}
%     0 & -i \\
%     i & 0
% \end{bmatrix}
% \]

% \subsection{Bramka Pauliego-Z}
% $|0\rangle = |0\rangle$ \\
% $|1\rangle = -|1\rangle$
% \[Z =
% \begin{bmatrix}
%     1 & 0 \\
%     0 & -1
% \end{bmatrix}
% \]

% \subsection{Bramka S}
% $|0\rangle = |0\rangle$ \\
% $|1\rangle = i|1\rangle$
% \[S =
% \begin{bmatrix}
%     1 & 0 \\
%     0 & i
% \end{bmatrix}
% \]

% \subsection{Bramka Fazy (T)}
% $|0\rangle = |0\rangle$ \\
% $|1\rangle = e^{i\frac{\pi}{4}}|1\rangle$
% \[T =
% \begin{bmatrix}
%     1 & 0 \\
%     0 & e^{i\frac{\pi}{4}}
% \end{bmatrix}
% \]

% \begin{equation*}
%     \Qcircuit @C=0.5em @R=0.0em @!R {
%                 \lstick{q0_{0}: \ket{0}} & \gate{H} & \qw & \ctrl{1} & \meter & \qw & \qw & \qw\\
%                 \lstick{q0_{1}: \ket{0}} & \gate{X} & \gate{H} & \targ & \qw & \meter & \qw & \qw\\
%                 \lstick{q0_{2}: \ket{0}} & \gate{H} & \meter & \qw & \qw & \qw & \qw & \qw\\
%                 \lstick{c0_{0}: 0} & \cw & \cw & \cw & \cw \cwx[-3] & \cw & \cw & \cw\\
%                 \lstick{c0_{1}: 0} & \cw & \cw & \cw & \cw & \cw \cwx[-3] & \cw & \cw\\
%                 \lstick{c0_{2}: 0} & \cw & \cw \cwx[-3] & \cw & \cw & \cw & \cw & \cw\\
%          }
% \end{equation*}
% \end{document}

% \documentclass[preview]{standalone}
% If the image is too large to fit on this documentclass use
\documentclass[draft]{beamer}
% img_width = 6, img_depth = 7
\usepackage[size=custom,height=10,width=13,scale=0.7]{beamerposter}
% instead and customize the height and width (in cm) to fit.
% Large images may run out of memory quickly.
% To fix this use the LuaLaTeX compiler, which dynamically
% allocates memory.
\usepackage[braket, qm]{qcircuit}
\usepackage{amsmath}
\pdfmapfile{+sansmathaccent.map}
% \usepackage[landscape]{geometry}
% Comment out the above line if using the beamer documentclass.
\begin{document}
\begin{equation*}
    \Qcircuit @C=0.5em @R=0.0em @!R {
                \lstick{q0_{0}: \ket{0}} & \gate{H} & \qw & \ctrl{1} & \meter & \qw & \qw & \qw\\
                \lstick{q0_{1}: \ket{0}} & \gate{X} & \gate{H} & \targ & \qw & \meter & \qw & \qw\\
                \lstick{q0_{2}: \ket{0}} & \gate{H} & \meter & \qw & \qw & \qw & \qw & \qw\\
                \lstick{c0_{0}: 0} & \cw & \cw & \cw & \cw \cwx[-3] & \cw & \cw & \cw\\
                \lstick{c0_{1}: 0} & \cw & \cw & \cw & \cw & \cw \cwx[-3] & \cw & \cw\\
                \lstick{c0_{2}: 0} & \cw & \cw \cwx[-3] & \cw & \cw & \cw & \cw & \cw\\
         }
\end{equation*}

\end{document}
